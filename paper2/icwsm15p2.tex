\def\year{2015}
\documentclass[letterpaper]{article}
\usepackage{aaai}
\usepackage{times}
\usepackage{helvet}
\usepackage{courier}

\frenchspacing
\setlength{\pdfpagewidth}{8.5in}
\setlength{\pdfpageheight}{11in}
\pdfinfo{
/Title (Modelling Online Bad Behaviour using Big Social Data)
/Author (Giles Oatley, Tom Crick)
/Keywords (Big Social Data, Online Behaviour Modelling, Complex Societal Interaction Modelling, Unnatural Language)
}
\setcounter{secnumdepth}{0}  

\begin{document}

\title{Modelling Online Bad Behaviour using Big Social Data}
\author{Giles Oatley \and Tom Crick\\
Department of Computing \& Information Systems\\
Cardiff Metropolitan University\\
Cardiff, UK
}

\maketitle

\begin{abstract}
\begin{quote}
The rise of Web 2.0 and social networking has facilitated the
publishing of user-generated content on an exponential scale; its
analysis is becoming increasingly important (and applicable) to the
empirical study of society (and societal change). However, this ``Big
Social Data'' from social media platforms differs significantly from
more traditional/formal sources.  In this paper we focus our attention
on modelling bad behaviour using big social data, for example
so-called unnatural language with its poor language construction but
also context dependent acronyms, jargon, ``leetspeak'' and
profanity. We include also in our discussion a study on language use
related to pornography, as well as the criminal element of identity
theft, fraud and deliberate deception.

Our long-term research goal is the development of complex (and robust)
behavioural modelling and profiling using a multitude of online
datasets; in this paper we review relevant tools for use in big social
data, and also present some novel ideas around a scalable solution to
swear words.
\end{quote}
\end{abstract}


\bibliographystyle{aaai}
\bibliography{icwsm15p2}

\end{document}
