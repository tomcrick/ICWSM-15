\def\year{2015}
%File: formatting-instruction.tex
\documentclass[letterpaper]{article}
\usepackage{aaai}
\usepackage{times}
\usepackage{helvet}
\usepackage{courier}
\usepackage{amsmath}
\usepackage[hyphens]{url}
\usepackage{tabularx,ragged2e}
\usepackage{graphicx}

\frenchspacing
\setlength{\pdfpagewidth}{8.5in}
\setlength{\pdfpageheight}{11in}
\pdfinfo{
/Title (Profiling Complex Online Interactions: What behaviour can you infer from a digital footprint?)
/Author (Mohamed Mostafa, Tom Crick, Giles Oatley)}
\setcounter{secnumdepth}{0}  
 \begin{document}
% The file aaai.sty is the style file for AAAI Press 
% proceedings, working notes, and technical reports.
%
\title{Profiling Complex Online Interactions:\\What behaviour can you infer from a digital footprint?}
\author{Mohamed Mostafa \and Tom Crick \and Giles Oatley\\
Department of Computing \& Information Systems\\
Cardiff Metropolitan University\\
Cardiff, UK
}
\maketitle
\begin{abstract}
\begin{quote}
This is an initial exploration of how people interact with complex
online information systems, using a wide range of digital data and
employing a range of personality and behavioural representation
systems. The information system is an online portal for submitting
applications for an international academic mobility grant scheme, and
we analyse document uploading behaviour, motivation letters, and
social media interactions.

We apply psycholinguistic techniques to the motivation letters to
determine the Five Factors personality scorings, and apply the same
techniques to interactions with a dedicated Facebook ``help''
page. The relative rankings are compared using the Kendall rank
correlation statistic.

Finally, we examine the upload footprint for the users and determine
several classes of behaviour. These again are compared against the
Five Factors, and in turn against the eventual grant status of the
applicant.
\end{quote}
\end{abstract}


\section{Introduction}


\bibliographystyle{aaai}
\bibliography{icwsm15p1}

\end{document}
