\def\year{2015}
\documentclass[letterpaper]{article}
\usepackage{aaai}
\usepackage{times}
\usepackage{helvet}
\usepackage{courier}
\usepackage[hyphens]{url}
\usepackage{paralist}
\usepackage{tabularx,ragged2e}
\usepackage{multirow}
\usepackage{float}
\usepackage{graphicx}
\usepackage{enumerate}
\usepackage[multiple]{footmisc}

% fix table columns
\newcolumntype{Y}{>{\centering\arraybackslash}X}

\frenchspacing
\setlength{\pdfpagewidth}{8.5in}
\setlength{\pdfpageheight}{11in}
\pdfinfo{
/Title (Modelling Online Bad Behaviour using Big Social Data)
/Author (Giles Oatley, Tom Crick)
/Keywords (Big Social Data, Online Behaviour Modelling, Complex Societal Interaction Modelling, Unnatural Language)
}
\setcounter{secnumdepth}{0}  

\begin{document}

\title{Modelling Online Bad Behaviour using Big Social Data}
\author{Giles Oatley \and Tom Crick\\
Department of Computing \& Information Systems\\
Cardiff Metropolitan University\\
Cardiff, UK
}

\maketitle

\begin{abstract}
\begin{quote}
The rise of Web 2.0 and social networking has facilitated the
publishing of user-generated content on an exponential scale; its
analysis is becoming increasingly important (and applicable) to the
empirical study of society (and societal change). However, this ``Big
Social Data'' from social media platforms differs significantly from
more traditional/formal sources.  In this paper we focus our attention
on modelling bad behaviour using big social data, for example
so-called unnatural language with its poor language construction but
also context dependent acronyms, jargon, ``leetspeak'' and
profanity. We also include a discussion on language use related to
pornography, as well as the criminal element of identity theft, fraud
and deliberate deception.

Our long-term research goal is the development of complex (and robust)
behavioural modelling and profiling using a multitude of online
datasets; in this paper we review relevant tools for use in big social
data, and also present some novel ideas around a scalable solution to
swear words.
\end{quote}
\end{abstract}


\section{Introduction}

There is a pressing need to develop new and robust techniques for
modelling online behaviour and identity, with the recent Snowden
incident~\cite{greenwald:2014} bringing significant attention to
profiling insider threats. User profiles or user models are vital in
many areas; we discuss our ideas relating to profiling complex
behaviours elsewhere~\cite{oatley+crick:2014}.  Examples of areas
where it is essential to obtain knowledge about users of software
applications include intelligent agents, adaptive systems, intelligent
tutoring systems, recommender systems, intelligent e-commerce
applications and knowledge management
systems~\cite{schiaffino+amandi:2009}.

Big data from social media platforms is now being used for a multitude
of purposes, including the obvious advertising, marketing and revenue
generation, as well as increasingly for government monitoring of
citizens\footnote{Twitter Transparency Report 2014:
\url{https://transparency.twitter.com/}}\footnote{Facebook Global
Government Requests Report
2014:\\\url{https://govtrequests.facebook.com/}}\footnote{Google
Transparency Report
2014:\\\url{http://www.google.co.uk/transparencyreport/}}, along with
covert security, intelligence community and military user profiling.
However, the publishing of user-generated content on this epic scale
has significantly changed qualitative and quantitative social
research, with its analysis becoming incredibly important to the
empirical study of society. There are interesting sociological uses of
studying or mining big social data, for instance exploring
cyber-physical crowds using location-based social networks or the
study of personality with large-scale benchmark social datasets and
corpora.

Big data from social networking sources differs significantly from
more formal sources.  With the advent of the social web, for instance
social networks, blogs, gaming, shopping and review sites, there is
now orders of magnitude more data available relating to uncensored
natural language, requiring the development of new techniques that can
meaningful analyse it. This uncensored language is rich in `unnatural'
language phenomena (as opposed to `natural' language, used in
traditional published media such as books and newspapers), defined as
``{\emph{informal expressions, variations, spelling errors...irregular
proper nouns, emoticons, unknown words}}''\footnote{2nd Unnatural
Language Processing Contest, part of the 17th Annual Meeting of the
Association for Natural Language Processing (NLP2011):
\url{http://www.anlp.jp/nlp2011/}}; it is also rich in
context-dependent acronyms, jargon, ``leetspeak'' and swear
words. Leet (or ``1337''), also known as eleet or leetspeak, is an
alternative alphabet for the English language that is used primarily
on the Internet and in geek/cyber communities. It uses various
combinations of ASCII characters to replace Latin script. For example,
leet spellings of the word leet include {\emph{1337}} and
{\emph{l33t}}; eleet may be spelled {\emph{31337}} or
{\emph{3l33t}}. See
\citeauthor{perea-et-al:2008}~\shortcite{perea-et-al:2008} for an
discussion of leet from a cognitive processing perspective.


\subsection{Representing complex behaviour and personality}

Advances in psychology research have suggested it is possible for
personality to be determined from digital
data~\cite{pennebaker+king:1999,vazire+gosling:2004,iacobelli-et-al:2011}.
Recent studies in the USA have suggested certain keywords and phrases
can signal underlying tendencies and that this can form the basis of
identifying certain aspects of
personality~\cite{woodworth-et-al:2012}.  Extrapolating forward
suggests that by investigation of an individual's online comments it
may be possible to identify individual's personality traits. Initial
evidence in support of this hypothesis was demonstrated in 2012 by
analysis of Twitter data for indicators of psychotic
behaviour~\cite{sumner-et-al:2012}. While in the past this has mainly
been the textual information contained in blogs, status posts and
photo comments~\cite{blamey-et-al-2012}, there is also a wealth of
information in the other ways of interacting with online
artefacts. For instance, it is possible to observe the
ordering/timings of button clicks of a user. Several authors have
looked at personality prediction (e.g. Five Factor personality traits)
based on information in a user's Facebook
profile~\cite{back-et-al:2010,golbeck-et-al:2001} and
speech~\cite{chung+pennebaker:2007,tausczik+pennebaker:2010}, as well
as also demonstrating significant correlations with fine affect
(emotion) categories such as that of excitement, guilt, yearning, and
admiration~\cite{mohammad+kiritchenko:2013}. There are also several
strands of related work based on the benchmark myPersonality
Project\footnote{\url{http://mypersonality.org/}}
dataset~\cite{celli-et-al:2013}, providing a medium for well-needed
comparative studies.

\citeauthor{mairesse-et-al:2007}~\shortcite{mairesse-et-al:2007}
highlighted the use of features from psycholinguistic databases
LIWC~\cite{pennebaker-et-al:2001} and MRC~\cite{wilson:1988} to create
a range of statistical models for each of the Five Factor personality
traits~\cite{norman:1963,peabody+goldberg:1989}.

In previous work~\cite{oatley+crick:2014} we utilised these methods to
develop a complex behavioural profile that included `two faces'; to
model that we can have several different modes of operation (ego
states). We performed our Five Factor analysis, and elaborated two
sets of Five Factor results for each user. We chose Chernoff
faces~\cite{chernoff:1973} for the visual appeal. The Five Factors are displayed as five
features on a stylised face, where:

\begin{itemize}
\item Width of hair represents {\emph{Conscientiousness}}; 
\item Width of eyes represents {\emph{Agreeableness}};
\item Width of nose represents {\emph{Openness to experience}}; 
\item Width of mouth represents {\emph{Emotional stability}};
\item Height of face represents {\emph{Extraversion}}.
\end{itemize}

\begin{figure}[!htb]
\centering
\includegraphics[width=\columnwidth]{images/twofaces.jpg}
\caption{Two faces for the same person. Each face represents a particular personality profile based on the Five Factors model}
\label{fig:twofaces}
\end{figure}

It should be noted that while researchers have continued to work with
the Five Factors model, there are well known
limitations~\cite{eysenck:1992,paunonen+jackson:2000,block:2010} that
are often overlooked by researchers. In particular, it has been
criticised for its limited scope, methodology and the absence of an
underlying theory. Attempts to replicate the Big Five in other
countries with local dictionaries have succeeded in some countries but
not in others~\cite{szirmak+deraad:1994,defruyt-et-al:2004}. While
\citeauthor{costa+mccrae:1992}~\shortcite{costa+mccrae:1992} claim
that their Five Factors model ``represents basic dimensions of
personality'', psychologists have identified important trait models,
for instance Cattell's 16 Personality Factors~\cite{cattell:1946} and
Eysenck's biologically based theory~\cite{eysenck:1947}.


\subsection{Readability and general measurements of English text}

There are several well-known measures of a text's readability, for
instance the Gunning Fog index~\cite{gunning:1968} and the
Flesch-Kincaid tests~\cite{kincaid-et-al:1975}. These can also be used
to profile an individual.

The Gunning Fog index indicates the number of years of formal
education a reader of average intelligence would need to understand
the text on the first reading and computes a scale
(unreadable/difficult/ideal/acceptable/childish) based on the ratios
of the counts of words to sentences and counts of complex words to
words. The Flesch-Kincaid tests consider word length (numbers of
syllables) and sentence length (numbers of words).


\section{Fraud, deception, identity theft and bad behaviour}

Automatically detecting a fraudulent account in a software system is a
very challenging task, requiring that a computer program grasps the
meaning of a persons' social data without significant human
insight. Consider the complexities of detecting specific events, for
instance marriage, anniversaries, payday, and so on, which are very
difficult features to represent and compute. This would be necessary
to start to `understand' what was happening or being represented in
someone's digital footprint. Perhaps prior to a marriage there will be
a list of potential words used in their social data, and a different
set of words used afterwards, and we can thus triangulate on our best
guess (obviously here we are ignoring direct knowledge of the date of
the event, for instance through calendars, or time-stamped photos or
albums tagged with `wedding' and so on). To facilitate this, a
significant amount of knowledge will need to be represented and
organised, for instance in specific knowledge modelling or bespoke
parsers and lexicons.

There is no doubt that profiling is very difficult, and in relation to
detecting (Five Factors) personality,
\citeauthor{mairesse:2013}~\shortcite{mairesse:2013} recommends
learning to map user inputs to elements that can improve its
interaction directly, that is, a sharply focused mode, and also that
there should just be a standard machine learning pipeline, and not a
`personality-based' user model. Researchers from the knowledge
representation/engineering community will recognise these problems --
the bottleneck of acquiring and appropriately representing domain
knowledge -- and this perhaps received its most practical explanation
in \citeauthor{richter:2003}'s~\shortcite{richter:2003} concept of
knowledge containers in case-based reasoning systems.

Consider the conclusions of the {\emph{Uncovering plagiarism,
authorship and social software misuse}} (PAN) stream of CLEF
2012\footnote{See:
\url{http://www.clef-initiative.eu/edition/clef2012/working-notes}} on
detection of sexual predators within chat
forums~~\cite{inches+crestani:2012}. {\emph{Porn Predators}} was an
evaluation lab (Sexual Predator Identification Competition) within a
broader annual conference covering plagiarism, authorship and social
software misuse. In relation to profiling technologies the conference
covers author identification and verification and author profiling,
including methods to answer the question whether two given documents
have the same author or not, and also predicting an author's
demographics. The tasks within {\emph{Porn Predators}} were to:

\begin{enumerate}[(i)]
\item identify the predators among all the users conversations;
\item identify the specific part of the conversations most distinctive
  of predator behaviour.
\end{enumerate}

In summary, for the first problem lexical and behavioural features
should be used, although there is no unique method proposed, and for
the second problem the most effective methods were those based on
filtering on a dictionary or lexicon, partly due to the lack of ground
truth for this specific problem.

The idea behind detecting fraud, or an insider threat, is that the
profile of an individual changes in significant ways, and by
developing a (knowledge) representational system rich enough we will
be able to detect this. And so we currently use a range of
psycholinguistic features, more complex behavioural or trait features,
lying/deception, and linguistic styles, formality, readability and so
on. Of course, there are simpler methods to detect bad behaviour
related to fake accounts, for instance we might be interested in
screening for non-UK cities, or the age of social networking
accounts. While it is easy to automatically generate social networking
profiles with followers and so on, and a range of links can be
developed, certain features like the age of posts and photo uploads
cannot be backdated to appear more established and credible.


\section{Rude words: a pornography study}

Colleagues carried out a research project investigating opinions on a
range of topics related to pornography usage; a web-based
questionnaire received over five thousand respondents
({\emph{n}}=5490). Several of the questions were open-ended, for instance how
the person became involved with the subject of pornography, their
particular interests and so on, eliciting on occasion very long
answers (c.2000 words). From the initial
findings~\cite{smith-et-al:2013}, the data is ill-structured, with
frequent usage of bad grammar and contains a large number of jargon
(swear) words relating to pornography and sexuality.

An aim of the original study was the investigation of the usage of
fantasy. This resonated with our general interest in determining
behaviour from data, and so we decided to explore the language
characteristics of the answers related specifically with fantasy. We
analysed the respondents text using the psycholinguistic databases
LIWC and MRC. The Dictionary of Affect in Language
(DAL)~\cite{sweeney+whissell:1984} was also planned to be used, due to
its specific uses for imagery-based language. We also used methods
derived from LIWC and MRC to determine personality traits and measures
such as formality and deception. We also wanted to get a general feel
for the level of the text, and to also see if there were any
correlations between literacy and readability.

Initially we focused on the specific questions that might reveal
something about the role of fantasy. For instance, among the many
options for the question ``{\emph{What are your reasons for looking at
pornography?}}'', among the list were the following:

\begin{itemize}
\item ``{\emph{A. To see things I might do}}'';
\item ``{\emph{B. To see things I can't do}}'';
\item ``{\emph{C. To see things I wouldn't do}}'';
\item ``{\emph{D. To see things I shouldn't do}}''. 
\end{itemize}

The `can't' and `wouldn't' choices clearly indicate respondents
utilising pornography more strongly as a form of fantasy. For this we
explored the Five Factors personality traits, in particular expecting
some correlation with the {\emph{Openness to Experience}} factor.

\begin{table}[!htb]
\centering
\begin{tabularx}{\columnwidth}{l Y Y Y Y}
%\begin{tabular}{c c c} 
\hline
& A & B & C & D\\ 
\hline
A & 1 &  & & \\
B & -0.72974 & 1 & & \\
C & -0.46635 & -0.06469 & 1 & \\
D & -0.33821 & 0.08321 & 0.091183 & 1\\
\hline
%\end{tabular}
\end{tabularx}
\caption{Correlation between question items. Where: A=``{\emph{To
see things I might do}}''; B=``{\emph{To see things I can't do}}''; C=
``{\emph{To see things I wouldn't do}}'' D=``{\emph{To see
things I shouldn't do}}''}
\label{tbl:abcd}
\end{table}

Analysis is ongoing, with the results to be published in the near
future; however there does appear to be a strong negative correlation
between participants who chose ``{\emph{A. To see things I might
do}}'' versus ``{\emph{B. To see things I can't do}}'', as originally
hypothesised. What was less convincing was our analysis of the Five
Factors, and we put this down to the measures we used
from~\cite{mairesse-et-al:2007} being derived from a very different
corpus. We are currently concentrating on the lower level features
from LIWC, MRC and DAL.

\begin{figure}[H]
\centering
\includegraphics[width=\columnwidth]{images/openA.jpg}
\caption{Openness to experience for A(y) (dotted) versus non-A (dashed)}
\label{fig:openA}
\end{figure}

\begin{figure}[H]
\centering
\includegraphics[width=\columnwidth]{images/openB.jpg}
\caption{Openness to experience for B(y) (dotted) versus non-A (dashed)}
\label{fig:openB}
\end{figure}

\begin{figure}[H]
\centering
\includegraphics[width=\columnwidth]{images/openC.jpg}
\caption{Openness to experience for C(y) (dotted) versus non-A (dashed)}
\label{fig:openC}
\end{figure}

\begin{figure}[H]
\centering
\includegraphics[width=\columnwidth]{images/openD.jpg}
\caption{Openness to experience for D(y) (dotted) versus non-A (dashed)}
\label{fig:openD}
\end{figure}

Among the categories of pornography for ``{\emph{What kinds of
sexually explicit materials do you access}}'' were the options
``Fiction sites'', ``Sex Blogs'' and
``Stories''. Table~\ref{tbl:textrel} shows the correlation between
these categories. All of these pornography sites are fiction or
story-based, and we examined the literary features of the respondents'
replies to see if this group of respondents write in a more
sophisticated way.

\begin{table}[!h]
\centering
\begin{tabularx}{\columnwidth}{l Y Y Y}
%\begin{tabular}{c c c} 
\hline
& Fiction & Blogs & Stories\\ 
\hline
Fiction & 1 & &\\
Blogs & 0.208086 & 1 & \\
Stories & 0.192054 & 0.041866 & 1\\
\hline
%\end{tabular}
\end{tabularx}
\caption{Correlation between text-related options.}
\label{tbl:textrel}
\end{table}

\begin{figure}[!h]
\centering
\includegraphics[width=\columnwidth]{images/lit-flesch.jpg}
\caption{Flesch measure. Fiction readers (dotted) versus non-fiction readers (dashed)}
\label{fig:flesch}
\end{figure}

\begin{figure}[!h]
\centering
\includegraphics[width=\columnwidth]{images/lit-fog.jpg}
\caption{Fog index. Fiction readers (dotted) versus non-fiction readers (dashed)}
\label{fig:fog}
\end{figure}

\begin{figure}[!htb]
\centering
\includegraphics[width=\columnwidth]{images/lit-kincaid.jpg}
\caption{Kincaid measure. Fiction readers (dotted) versus non-fiction readers (dashed)}
\label{fig:kincaid}
\end{figure}

\begin{figure}[!h]
\centering
\includegraphics[width=\columnwidth]{images/lit-percentcomplexwords.jpg}
\caption{Percent complex words. Fiction readers (dotted) versus non-fiction readers (dashed)}
\label{fig:percentcomplex}
\end{figure}

\begin{figure}[!h]
\centering
\includegraphics[width=\columnwidth]{images/lit-syllablesperword.jpg}
\caption{Syllables per word. Fiction readers (dotted) versus non-fiction readers (dashed)}
\label{fig:syllables}
\end{figure}

\begin{figure}[!h]
\centering
\includegraphics[width=\columnwidth]{images/lit-wordspersentence.jpg}
\caption{Words per sentence. Fiction readers (dotted) versus non-fiction readers (dashed)}
\label{fig:wordpersent}
\end{figure}

\begin{figure}[!h]
\centering
\includegraphics[width=\columnwidth]{images/lit-wordcount.jpg}
\caption{Word count. Fiction readers (dotted) versus non-fiction readers (dashed)}
\label{fig:wordcount}
\end{figure}

In all cases (see Figures~\ref{fig:flesch}--\ref{fig:wordcount}), it
seems that the users favouring the text/story-based pornography have
higher values than their non-text contemporaries.


\section{Disambiguating profanity}

WordNet\footnote{\url{http://wordnet.princeton.edu/}} is a large
lexical database of English; nouns, verbs, adjectives and adverbs are
grouped into sets of cognitive synonyms (synsets), each expressing a
distinct concept, and each synset is interlinked by means of
conceptual-semantic and lexical relations. Words that are found in
close proximity to one another in the network are semantically
disambiguated. WordNet
Affect\footnote{\url{http://wndomains.fbk.eu/wnaffect.html}}, a
hierarchical set of emotional categories, and
SentiWordNet\footnote{\url{http://sentiwordnet.isti.cnr.it/}}, synets
are assigned sentiment scores (positivity, negativity, objectivity),
are built on top of WordNet.

\citeauthor{millwood-hargrave:2000}'s~\shortcite{millwood-hargrave:2000}
study for the UK regulator Ofcom (formerly, the Broadcasting Standards
Commission) was designed to test people’s attitudes to swearing and
offensive language, and to examine the degree to which context played
a role in their reactions. Included in the report was attitudes
towards swearing and offensive language `in life', including a range
of swear words and terms of abuse. Appendix 2’s `list of words'
contained positions of the top swear words (saying ``very severe'',
``fairly severe'', ``quite mild'' and ``not swearing'') and their
ranking from 1998 to 2000.

The study of swear words has a longstanding position in linguistics,
with the academic journal {\emph{Maledicta: The International Journal of
Verbal Aggression}} running from 1977 until 2005. Maledicta was
dedicated to the study of the origin, etymology, meaning, use and
influence of vulgar, obscene, aggressive, abusive and blasphemous
language. Unfortunately we do not have resources such as databases in
the literature; furthermore, WordNet does not contain the range of
swear words we encountered in our data and is no use for disambiguating
our text. Wikipedia, however, fared much better. But even better than
these were Roger's Profanisaurus and Urban Dictionary.

Roger's
Profanisaurus\footnote{\url{http://www.viz.co.uk/profanisaurus.html}}
is a lexicon of profane words and expressions; the 2005 version (the
Profanisaurus Rex), contains over 8,000 words and phrases, with a
further-expanded version released in 2007. Unlike a traditional
dictionary or thesaurus, the content is enlivened by often pungent or
politically incorrect observations and asides intended to provide
further comic effect.

Urban Dictionary\footnote{\url{http://www.urbandictionary.com/}} is a
Web-based dictionary that contains nearly eight million definitions as
of December 2014. Originally, Urban Dictionary was intended as a
peer-reviewed dictionary of slang or cultural words or phrases not
typically found in standard dictionaries, with words or phrases on
Urban Dictionary having multiple definitions, usage examples and tags.

We created different gazetteers related to rude words (available upon
request); one list was based on Wikipedia entries, and another on
lists from Urban Dictionary. The Wikipedia list was created from link
text on the Wikipedia porn sub-genre
page\footnote{\url{http://en.wikipedia.org/wiki/List_of_pornographic_sub-genres}}
(link ``anchor text'' is a typical approach in semantic relatedness
studies). This was comprised of 250 words. The Urban Dictionary list
was created from the ``sex''
category\footnote{\url{http://www.urbandictionary.com/category/sex}}
(by no means exhaustive -- it is a fraction of the pornography-related
terms in Urban Dictionary). This was comprised of 156 words. We
implemented two metrics for rude words, the key idea of which is to
have a simple mathematical model that enables us to estimate the
life-history value of a token.\\

\begin{enumerate}
\item {\textbf{IndexOfRudeWords:}} For each sentence we have an
  ordered set \{{\emph{rude1,rude2,rude3}}\}. This would be mapped to
  the following (rude words for sentences appearing sequentially);
  only the first mention is added to list.\\

\begin{verbatim}
    {
        rude1 -> .25
        rude2 -> .5
        rude3 -> .75
    }

\end{verbatim}

We could say that this represents the relative position in terms of rude word occurrence, and therefore we can call this a normalized rude word index.\newline

\item {\textbf{Sentence Index:}} For each sentence in a block of text, the index of
the sentence is used.\\

\begin{verbatim}
    {
        // rude1 mentioned 
        // in sentences 1 and 3.
        rude1 -> {1,3}   
        // etc...
        rude2 -> {2}
    }

\end{verbatim}
\end{enumerate}

For (1) and (2) we can compute an average metric for each rude word
over all sentences, additionally with the possibility of excluding
those below some frequency (per word) and/or excluding responses where
the number of rude words was very low. Another alternative would be
representing as a fractional token index.


\section{Conclusions and future work}

Existing NLP tools are known to struggle with unnatural language:
``{\emph{demonstrated that existing tools for POS tagging, chunking
and Named Entity Recognition perform quite poorly when applied to
tweets}}''~\cite{ritter-et-al:2011} and ``{\emph{showed that
[lengthening words] is a common phenomenon in
Twitter}}''~\cite{brody+diakopoulos:2011}, presenting a problem for
lexicon-based approaches. These investigations both employed some form
of inexact word matching to overcome the difficulties of unnatural
language. We have made no attempt to use inexact string matching or to
make use of a leetspeak parser. This will form part of future work.

To assist with the ongoing knowledge modelling problem in this domain
we recognise the need to utilise specific lexicons that keep pace with
the language used, for instance the use of Urban Dictionary to resolve
swear words. We need to study how in what precise manner this resource
keeps pace with popular culture.

There are many other lists of pornographic words, which we compiled
from miscellaneous sources; however, we are mainly interested in
sources such as Wikipedia and Urban Dictionary as these are maintained
by a similar community that uses the words in social networking. In this
way we do not have to concern ourselves about this knowledge
engineering process, merely concern ourselves about the representation
and quality of meaning or definitions. We will in future make use of
the voting scores available on Urban Dictionary, and look to
incorporate new resources such as Roger's Profanisaurus.

Future work will continue with the range of techniques to analyse
content for words/phrases associated with certain key
emotions/states. We will further develop the bottom-up use of two and
three word n-grams and top-down approach using the LIWC, MRC and DAL
databases~\cite{iacobelli-et-al:2011}.


\bibliographystyle{aaai}
\bibliography{icwsm15p2}

\end{document}
