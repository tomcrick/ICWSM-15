\def\year{2015}
\documentclass[letterpaper]{article}
\usepackage{aaai}
\usepackage{times}
\usepackage{helvet}
\usepackage{courier}
\usepackage{amsmath}
\usepackage[hyphens]{url}
\usepackage{tabularx,ragged2e}
\usepackage{multirow}
\usepackage{graphicx}

% fix table columns
\newcolumntype{Y}{>{\centering\arraybackslash}X}

\frenchspacing
\setlength{\pdfpagewidth}{8.5in}
\setlength{\pdfpageheight}{11in}
\pdfinfo{
/Title (Profiling Complex Online Interactions: What Behaviour Can You Infer From a Digital Footprint?)
/Keywords (Social Networks, Social Behaviour, Psycholinguistics, Social Computing, Data Mining)
}
\setcounter{secnumdepth}{0}  

\begin{document}

\title{Profiling Complex Online Interactions:\\What Behaviour Can You Infer From a Digital Footprint?}
\author{ICWSM-15 Paper 8}

\maketitle

\begin{abstract}
\begin{quote}
This is an initial exploration of how people interact with complex
online information systems, using a wide range of digital data and
employing a number of personality and behavioural representation
systems. The information system is an online application portal for an
international academic mobility grant scheme, and we analyse document
uploading behaviour, motivation letters and social media
interactions.

We apply psycholinguistic techniques to the motivation letters to
determine the Five Factors personality scorings, and apply the same
techniques to interactions with a dedicated Facebook page. The
relative rankings are compared using the Kendall rank correlation
statistic.

Finally, we examine the upload footprint for the users and determine
several classes of behaviour. These again are compared against the
Five Factors, and in turn against the eventual grant status of the
applicant.
\end{quote}
\end{abstract}


\section{Introduction}

Understanding how to develop software to facilitate its effective and
efficient use is a core part of the broad field of human-computer
interaction. Different users form different conceptual models about
their interactions and have different ways of obtaining and developing
knowledge and skills; cultural and national differences may also play
a significant role. Another consideration in human-computer
interaction is that technology -- and in particular, user interface
technology -- changes rapidly, offering new interaction possibilities
to which previous research findings may not necessarily
apply. Alongside this, user preferences (and the way in which they
interact with the software) change as they gradually master new
interfaces and environments.

Personality and behaviour can be determined from digital
data~\cite{pennebaker-et-al:2001,vazire+gosling:2004,iacobelli-et-al:2011,oatley+crick:2014};
while in the past this has mainly been the textual information
contained in blogs, status posts and photo
comments~\cite{blamey-et-al-2012,blamey-et-al-2013}, there is also a
wealth of information in the other ways of interacting with digital
artefacts. For instance, it is possible to observe the ordering (and
frequency) of button clicks for a user.

\citeauthor{mairesse-et-al:2007}~\shortcite{mairesse-et-al:2007}
demonstrate the use of features from the psycholinguistic databases
LIWC~\cite{pennebaker-et-al:2001} and MRC~\cite{wilson:1988} to create
a range of statistical models for each of the Five Factors personality
traits~\cite{norman:1963,peabody+goldberg:1989,goldberg:1990}. These
five traits are: {\emph{Extraversion}}, {\emph{Emotional Stability}},
{\emph{Agreeableness}}, {\emph{Conscientiousness}} and {\emph{Openness
to Experience}}. Equation~\ref{eqn:extraversion} describes
{\emph{Extraversion}}, where each feature is prefixed by the
containing database.

\begin{equation}\label{eqn:extraversion}
\small
\begin{aligned}
\verb!Extraversion =!\\
&\verb!-0.0379 * MRC.K_F_NSAMP +!\\
&\verb!-0.0803 * LIWC.UNIQUE +!\\
&\verb!-0.6074 * LIWC.ABBREVIATIONS +!\\
&\verb! 0.1445 * LIWC.PRONOUN +!\\
&\verb!-0.3941 * LIWC.HEARING +!\\  
&\verb!17.1407;!
\end{aligned}
\end{equation}

The test system analysed in this paper is the user interaction for a
web-based educational application portal: ``{\emph{Online Portal for
Scholarship Mobility}}''. We make use of textual data, analysing with
these same psycholinguistic techniques, and employ standard
statistical methods on non-textual data. The textual data also
includes interaction with a dedicated Facebook page dedicated to
resolving problems with the applications (of which there were many);
and, the actual documents submitted, including a free-text application
motivation letter. The non-textual data includes the final scoring of
the individual for the grant they applied for (e.g. success, reserved,
failure) and the individual's behaviour on the site (when they uploaded
their documents, how close to the deadline and so on).


\section{Domain Description and Available Data}

Our data comes from an online portal for a European Union (EU)
international scholarship mobility hosted at a UK university. The aim
of the mobility programme is to enhance quality in higher education
through scholarships and academic cooperation between Europe and the
rest of the world. It provides mobility grants for students at
different academic levels (Undergraduate, Masters, PhD, Post-Doctoral,
Faculty) and has numerous courses available from a wide range of
institutions across the EU.

The details of the call were as follows: there were 2,706 applications
submitted by 1,170 candidates, applying to 10 EU universities and 10
non-EU universities. The system allows an applicant to apply to up to
three courses from all courses offered by the 10 universities, and the
applicant is required to assign a priority for each course. This
priority field is used in the selection stage, for instance if the
applicant assigned Course A as priority (1) and Course B as priority
(2) and the applicant is accepted in both courses, then the first
priority will be offered.

Each mobility call has an opening date/time and closing date/time,
with occasional extensions given for specific reasons (for instance
due to administrative reasons or technical issues with the
portal). Applicants are required to submit for their application
certain mandatory files, such as motivation letter,
passport/identification, curriculum vitae), as well as optional files
(supporting documents). The primary modes of communication between
candidates and the project team is via emails, telephone and the
dedicated Facebook page. The selection process divided into three
stages: Eligibility, Evaluation and Final Selection.

\subsection{Eligibility}

The project team scans each document to make sure all documents are
uploaded correctly, for example users might upload an empty image in
place of motivation letters and these need to be manually
verified. There are mandatory documents to be uploaded to meet the
minimum requirements of the call.

\subsection{Evaluation}

Eligible users are evaluated by a technical expert from the host
university, based on the following: motivation letter; academic
merit/performance (qualifications/grades); English language
competencies e.g. International English Language Testing System
(IELTS)\footnote{\url{http://www.ielts.org/}} score; and
academic/research proposal.

\subsection{Final Selection}

Based on which rank the applicant assigned to the host university, the
final selection is the top {\emph{n}} of applicants. {\emph{n}} is
calculated based in the host capacity and budget of the project. This
process results in the final classification of the applicant as
either:

\begin{itemize}
\item {\emph{Accepted}} (ranked highest);
\item {\emph{Reserved}} (passed but not selected);
\item {\emph{Rejected}} (below passing grade); 
\item {\emph{Ineligible}} (missing documents or out-dated documents).
\end{itemize}


\section{Experimental Design}

Initially, we carried out three different types of experiments:
Experiment 1 compared the motivation letters against the Facebook
interactions; Experiment 2 compared the interaction footprint against
the motivation letters; Experiment 3 checks the raw data using
multiple regression.

\subsection{Experiment 1}

Applicants are required to upload a description of why they are
applying for this particular mobility grant, the motivation
letter. Applicants also communicated with the project team through the
project Facebook page.

We are interested in the similarity between language characteristics
used in the much more formal documents (for instance the motivation
letter) and that used in online social media (the Facebook page). The
Facebook interaction consists of posts, comments and ``likes''. We
extracted the text from all motivation letters and Facebook
interactions and analysed both blocks of text according to the Five
Factor personality traits as discussed previously.
 
After extracting the Five Factors for each applicant we compared the
relative positions of each of the Five Factors in both lists using the
Kendall rank correlation statistic~\cite{kendall:1938}. We did
this for: all applicants ({\emph{All}}); only the accepted applicants
({\emph{Accepted}}); only the reserved applicants ({\emph{Reserved}});
and only the rejected applicants ({\emph{Rejected}}).

The average Kendall's tau coefficient value is reported for these
groups, for each of the Five Factor features. By considering rank
position and not absolute value, we mitigate against explaining values
without baseline experimentation.

\subsection{Experiment 2}

We simplified an applicant's interaction, or timeline, with the portal
to include the following milestones: {\emph{T0}} Registration Time;
{\emph{T1}} First Action; {\emph{T2}} Last Action; and, {\emph{T3}}
Submission. Additionally we represented an extension to the submission
deadline as {\emph{T4}} Extension. In this way we can represent an
applicant's interaction as shown in Figure~\ref{fig:timelines}, which
shows seven example timelines.

\begin{figure}[!ht]
\centering
\includegraphics[width=\columnwidth]{images/legend.jpg}
\includegraphics[width=\columnwidth]{images/timelines.jpg}
\caption{Seven user timelines. {\emph{T0}} (black bar) is when the applicant first
registered with the call. {\emph{T1}} (red bar) represents when the applicant
uploaded their first document, or First Action. {\emph{T2}} (green bar)
represents an applicants' Last Action. {\emph{T3}} (blue bar) represents the
applicants' Submission. {\emph{T4}} (aquamarine bar) represents the first
deadline (certain calls had initial deadlines extended)}
\label{fig:timelines}
\end{figure}

Using these milestones we are able to identify interesting behaviours
that compare and contract with personality traits and other sources of
information. Behaviours such as: how long it was before an applicant
became aware of the call, and when they registered; how long after
registration did the applicant carry out their first action with
the system; how long did they take to complete their application; and,
how close to the deadline did they submit their application.

We divided the timeline of the call into five segments as presented in
the following Table~\ref{tbl:tlperiods}. The complete timeline from
opening to final close was 125 days. There was an extension from day
112 until day 125. The segments or timeline periods are determined as
percentage chunks of the total timeline, for instance segment
{\emph{S0}} is the first 20\% of the timeline, and so ranges from day
1 until day 25, segment {\emph{S1}} ranges from day 26 until day 50,
and so on.

\begin{table}[!ht]
\centering
\begin{tabularx}{\columnwidth}{Y Y Y}
%\begin{tabular}{c c c} 
\hline
Segment & Start & Finish  \\ 
\hline
{\emph{S0}} & 0 & 20\\
{\emph{S1}} & 20 & 40\\
{\emph{S2}} & 40 & 60\\
{\emph{S3}} & 60 & 90\\ 
{\emph{S4}} & 90 & 100\\ 
\hline
%\end{tabular}
\end{tabularx}
\caption{Timeline periods as percentages of total timeline}
\label{tbl:tlperiods}
\end{table}

Using these segments we were able to assign the various applicant
actions ({\emph{T0}} Registration, {\emph{T1}} First Upload,
{\emph{T2}} Last Upload, {\emph{T3}} Submission) to various time
periods. This allowed us to assign applicants to statistically
significant categories, and also to add in a few categories from
observations. These are shown in Table~\ref{tbl:apptlseg}; as you can
see, a small number of applicants ({\emph{n}}=4) registered within the
segment S1 (20-40\% of timeline), and then uploaded all of their
documents and submitted within the segment S3 (60-90\% of
timeline). This is represented by Class A, the first row; successive
rows can be interpreted in the same manner.

\begin{table}[!ht]
\centering
\begin{tabularx}{\columnwidth}{Y Y Y Y Y Y}
%\begin{tabular}{c c c} 
\hline
Class & {\emph{n}} & {\emph{T0}} & {\emph{T1}} & {\emph{T2}} & {\emph{T3}} \\ 
\hline
{\emph{A}} & 4 & S1 & S3 & S3 & S3\\
{\emph{B}} & 14 & S2 & S2 & S2 & S2\\
{\emph{C}} & 128 & S2 & S3 & S3 & S3\\
{\emph{D}} & 29 & S2 & S3 & S4 & S4\\
{\emph{E}} & 678 & S3 & S3 & S3 & S3\\
{\emph{F}} & 202 & S3 & S3 & S4 & S4\\
{\emph{G}} & 9 & S3 & S4 & S4 & S4\\
{\emph{H}} & 54 & S4 & S4 & S4 & S4\\
\hline
%\end{tabular}
\end{tabularx}
\caption{Applicants' timeline actions assigned to segments}
\label{tbl:apptlseg}
\end{table}

\begin{table*}[!ht]
\centering
\begin{tabularx}{\textwidth}{c X l}
%\begin{tabular}{c c c} 
\hline
Class & Description & Potential Alias  \\ 
\hline
{\emph{A}} & Register early, and take some time to upload documents, but submit with plenty of time before deadline & EverythingEarly\\
{\emph{B}} & Register reasonably early, but then upload documents and submit straight after with plenty of time before deadline, making no amendments & QuiteEarlyAndQuick\\
{\emph{C}} & Similar to Class B, but submitting more slowly & Cautious\\
{\emph{D}} & Registers reasonably early, and then takes time to upload, and only submits at the last days & VeryCautious\\
{\emph{E}} & Latecomer to registration, but then uploads and submits
quickly thereafter & Cautious\\
{\emph{F}} & Latecomer to registration, but then uploads and submits
slowly & Cautious\\
{\emph{G}} & Latecomer to registration, but delays uploading and
submission to last days & Cautious\\
{\emph{H}} & Does everything at the last days, from registration to submission & EverythingLastMinute\\
\hline
%\end{tabular}
\end{tabularx}
\caption{Description of each class}
\label{tbl:classdesc}
\end{table*}

We do not want to be too quick to ascribe an alias to the behaviours,
as we recognise that there are several possible interpretations;
nevertheless, we have used the `Potential Alias' column in
Table~\ref{tbl:classdesc} to indicate some initial thoughts.


\subsection{Experiment 3}

While the compound features of the Five Factors are an interesting
perspective, we also needed to check the raw data underneath this, in
the form of the psycholinguistic features LIWC and MRC. For this
investigation we chose multiple regression.

We extracted the LIWC row data features (87 features) from the
motivation letters and analysed the input data set against the
`status' of the application. The method used was multiple regression,
the dependent variable being `status' and the independent variables
are the LIWC features. We proceeded as follows:

\begin{itemize}
\item Our first assumption is multicollinearity, which refers to the
  relationship when two independents variables are highly correlated;
\item Removing the above features, carry out regression;
\item Detecting outliners using Mahalanobis distance, and since we
  have 60 features remaining after the multicollinearity elimination,
  our critical value is: {\textbf{99.607}} (see Table~\ref{tbl:sampledata});
\item Screening for outliners, since multiple regression is very
  sensitive regarding outliners;
\item Making sure that we have linear relationship between the
  independent variables and the outcome.
\end{itemize}


\section{Results}

The result presented in this section are an initial indication of our
research with this system; we are currently developing more detailed
experiments and research with this rich dataset.

\subsection{Experiment 1}

Table~\ref{tbl:avrankcorr} shows the results of the Kendall's tau
coefficient, specifically the variant that makes adjustments for ties
({\emph{Tau-b}}). Values of {\emph{Tau-b}} range from -1 (100\%
negative association, or perfect inversion) to +1 (100\% positive
association, or perfect agreement). A value of zero indicates the
absence of association.

\begin{table}[!ht]
\centering
\begin{tabularx}{\columnwidth}{l Y Y Y Y Y}
%\begin{tabular}{c c c} 
\hline
Group & E & ES & A & C & O \\ 
\hline
{\emph{All}} & $-0.094$ & 0.099 & 0.145 & 0.025 & {\textbf{$-0.379$}}\\
{\emph{Accepted}} & 0.000 & 0.000 & 0.000 & 0.000 & {\textbf{0.800}}\\
{\emph{Rejected}} & $-0.244$ & {\textbf{0.333}} & $-0.067$ & 0.022 & $-0.244$\\
{\emph{Reserved}} & 0.010 & 0.010 & 0.162 & 0.153 & {\textbf{$-0.6$}}\\
\hline
%\end{tabular}
\end{tabularx}
\caption{Average rank correlation for applicant group versus personality
trait (E: {\emph{Extraversion}}; ES: {\emph{Emotional Stability}}; A: {\emph{Agreeableness}}; C:
{\emph{Conscientiousness}}; O: {\emph{Openness to Experience}})}
\label{tbl:avrankcorr}
\end{table}


The most significant positive relationship is between those applicants
{\emph{Accepted}} and the feature {\emph{Openness to Experience}}
({\emph{Tau-b}} = 0.8). A strong negative relationship exists between
those applicants {\emph{Reserved}} and the feature {\emph{Openness to
Experience}} ({\emph{Tau-b}} = -0.6).

\subsection{Experiment 2}

The following Figures~\ref{fig:extraversion}--\ref{fig:openexp} show
box and whisper plots for each of the Five Factors, with the y-axis of
each figure displaying the range for that particular feature. For
example, Figure~\ref{fig:extraversion} displays the Extraversion
feature, and the y-axis displays these values accordingly. The x-axis
is comprised of the various classes from
Figure~\ref{fig:extraversion}, combined with the status of the
application (1. {\emph{Accepted}}, 2. {\emph{Rejected}},
3. {\emph{Reserved}}, 4. {\emph{Ineligible}}). Therefore, A1 are the
Class A applicants who were {\emph{Accepted}}, distinguished from A2,
who were the same class (i.e. same activity based on
timeline/milestones), but who were {\emph{Rejected}}.

\begin{figure*}[!hp]
\centering
\includegraphics[width=0.8\textwidth]{images/Extraversion_ver1.jpg}
\caption{{\emph{Extraversion}}. All features are hard to distinguish between, excepting that B2 is
significantly smaller than B3 and B4}
\label{fig:extraversion}
\end{figure*}

\begin{figure*}[!hp]
\centering
\includegraphics[width=0.8\textwidth]{images/EmotionalStability_ver1.jpg}
\caption{{\emph{Emotional Stability}}. No real features larger or smaller, although the range on all of the E
features seems much greater than the other features}
\label{fig:emotstab}
\end{figure*}

\begin{figure*}[!hp]
\centering
\includegraphics[width=0.8\textwidth]{images/Agreeableness_ver1.jpg}
\caption{{\emph{Agreeableness}}. D4 is significantly smaller than D1, D2, and D3. G2 appears
significantly less conscientious than G3}
\label{fig:agreeableness}
\end{figure*}

\begin{figure*}[!hp]
\centering
\includegraphics[width=0.8\textwidth]{images/Conscientiousness_ver1.jpg}
\caption{{\emph{Conscientiousness}}. G2 appears significantly less conscientious than G3. To a lesser
degree D4 is smaller than D1, D2, and D3}
\label{fig:conscientiousness}
\end{figure*}

\begin{figure*}[!hp]
\centering
\includegraphics[width=0.8\textwidth]{images/OpennessToExperience_ver1.jpg}
\caption{{\emph{Openness to Experience}}. As with Emotional Stability, there are no exceptional features,
although the range on all of the E features seems much greater than
the other features. The Class E were the applicants that were relative
late comers to registration, but who then uploaded and submitted
quickly thereafter. Openness to Experience would seem to have very
little relationship with this class of applicant}
\label{fig:openexp}
\end{figure*}

\subsection{Experiment 3}

We extracted the LIWC row data features (87 features) from the
motivation letters and analysed the input data set against the
`status' of the application. The method used was multiple regression,
the dependent variable being `status' and the independent variables
are the LIWC features (see Table~\ref{tbl:sampledata}).

\begin{table}[!ht]
\centering
\begin{tabularx}{\columnwidth}{l Y Y Y Y Y}
%\begin{tabular}{c c c} 
\hline
UID & WC & WPS & UNIQUE & SIXLTR\\ 
\hline
1003 & 364 & 24.2667 & 41.4835 & 37.9121\\
1008 & 275 & 22.9167 & 61.4545 & 22.5455\\
1010 & 197 & 8.20833 & 68.0203 & 37.0558\\
1014 & 577 & 19.2333 & 53.7262 & 28.9428\\
1016 & 348 & 19.3333 & 55.4598 & 29.5977\\
1023 & 538 & 16.8125 & 53.9033 & 26.9517\\
1033 & 517 & 23.5 & 54.352 & 35.9768\\
1035 & 165 & 23.5714 & 62.4242 & 27.8788\\
1039 & 388 & 16.1667 & 56.1856 & 31.701\\
1040 & 491 & 14.8788 & 58.2485 & 33.4012\\
1049 & 462 & 25.6667 & 55.8442 & 33.1169\\
1058 & 293 & 32.5556 & 55.2901 & 26.9625\\
1069 & 436 & 29.0667 & 52.5229 & 26.1468\\
1073 & 162 & 27 & 61.1111 & 25.9259\\
1078 & 334 & 17.5789 & 55.988 & 34.4311\\
\hline
%\end{tabular}
\end{tabularx}
\caption{Sample of the data set: we have 87 LIWC features and more than
  1000 candidates. UID represent the user and rest of the columns represent
  the LIWC features}
\label{tbl:sampledata}
\end{table}

\subsection{Result Set}

The first result set shows the correlation between dependent variable
(status) and independents variables (LIWC features). In our first
assumption multicollinearity we use an R value of 0.7 or higher to say
two predictable values have multicollinearity. Based on
Table~\ref{tbl:sampledata}, if the tolerance is smaller than 0.1 then
we have probability of multicollinearity, while VIF is the inverse of
the tolerance, and so in the case of VIF greater than 10 then we have
a case of multicollinearity. In this way, we found that the below LIWC
features are related through multicollinearity:

\begin{verbatim}
    REFERENCE_PEOPLE      SEEING
    LEISURE_ACTIVITY      SELF
    AFFECTIVE_PROCESS     WE
    PHYSICAL_STATES       MUSIC
    POSITIVE_EMOTION      TV_OR_MOVIE
    SPORTS                FEELING
    OTHER                 SLEEPING
    BODY_STATES           HEARING
    YOU                   SEXUALITY
    SENSORY_PROCESS       OCCUPATION
    HOME                  SOCIAL_PROCESS
    PRONOUN               COGNITIVE_PROCESS
    DIC                   NEGATIVE_EMOTION
\end{verbatim}

\begin{table}[!h]
\begin{tabularx}{\columnwidth}{l l Y Y}
%\multicolumn{2}{l}{Coefficients}            &                       &\\
\hline
\multicolumn{2}{l}{\multirow{2}{*}{Model}} &
\multicolumn{2}{c}{{\emph{Collinearity Statistics}}} \\
\multicolumn{2}{c}{}                        & Tolerance             & VIF                \\
\hline
1   & (Constant)           & -                     & -                  \\
& REFERENCE\_PEOPLE    & 0.003                 & 290.011            \\
& LEISURE\_ACTIVITY    & 0.004                 & 266.927            \\
& AFFECTIVE\_PROCESS   & 0.006                 & 181.557      \\
& PHYSICAL\_STATES	& 0.006	& 181.512 \\
& POSITIVE\_EMOTION	& 0.006	& 176.569  \\
& SPORTS	& 0.007	& 139.147 \\
& OTHER	& 0.007	& 138.914 \\
& BODY\_STATES	& 0.008	& 122.386 \\
& YOU	& 0.009	& 108.779 \\
& SENSORY\_PROCESS	& 0.011	& 89.824 \\
& HOME	& 0.015	& 67.555 \\
& PRONOUN	& 0.022	& 46.061 \\
& DIC	& 0.023	& 43.246 \\
& SEEING	& 0.025	& 39.690 \\
& SELF	& 0.029	& 34.103 \\
& WE	& 0.033	& 30.494 \\
& MUSIC	& 0.037	& 27.254 \\
& TV\_OR\_MOVIE	& 0.041	& 24.260 \\
& FEELING	& 0.043	& 23.210 \\
& SLEEPING	& 0.044	& 22.841 \\
& HEARING	& 0.047	& 21.488 \\
& SEXUALITY	& 0.049	& 20.536 \\
& OCCUPATION	& 0.050	& 19.857 \\
& SOCIAL\_PROCESS	& 0.070	& 14.252 \\
& COGNITIVE\_PROCESS	& 0.075	& 13.407 \\
& NEGATIVE\_EMOTION	& 0.089	& 11.213 \\
\hline
\end{tabularx}
\caption{Coefficients of multicollinearity variance influence factor}
\label{tbl:coefficmulti}
\end{table}

\subsection{Mahalanobis Distance}

\begin{figure}[!h]
\centering
\includegraphics[width=\columnwidth]{images/fig-x-1.jpg}
\caption{Normal P-P Plot}
\label{fig:normalpp}
\end{figure}

\begin{figure}[!h]
\centering
\includegraphics[width=\columnwidth]{images/fig-x-2.jpg}
\caption{Scatterplot}
\label{fig:scatterplot}
\end{figure}

Checking Mahalanobis distance, we found that 102 records exceed the
critical values, so we have removed these records since it is more
than 2\% of the total number. 

\begin{table}[!h]
\begin{tabularx}{\columnwidth}{Y Y Y Y Y}
%\multicolumn{5}{l}{Model Summary} \\
\hline
Model & R & R Sq. & Adj. R Sq. & Std. Err of Est.\\
\hline
1   & .262$^a$ & 0.069  & 0.010 & 0.796\\
\hline
\end{tabularx}
\caption{Model summary after removing the multicollinearity features
  and above critical value of Mahalanobis distance}
\label{tbl:modelsum}
\end{table}

\begin{table}[!h]
\begin{tabularx}{\columnwidth}{Y Y Y Y Y Y}
\hline
 & Sum of Sq. & df & Mean Sq. & F & Sig.\\
\hline
Regress. & 41.602 & 56 & 0.743 & 1.172 & 0.187$^b$\\
Residual  & 563.465 & 889  & 0.634 & - & -\\
Total & 605.067 & 945 & - & - & -\\
\hline
\end{tabularx}
\caption{Evaluation of the model and ability to predicate the status values}
\label{tbl:anova}
\end{table}


\begin{table}[!h]
\begin{tabularx}{\columnwidth}{l l Y Y}
\hline
Model & & Stnd. Coeff. & Sig.\\
\hline
& & Beta & \\
1 & (Constant) & & 0.000\\
 & NEGATIONS & 0.109 & 0.004\\
& QMARK	& 0.107	& 0.199 \\
& SPACE	& 0.098	& 0.044 \\
& ABBREVIATIONS	& 0.076	& 0.038 \\
& CAUSATION	& 0.073	& 0.051 \\
& DASH	& 0.068	& 0.093 \\
& UNIQUE	& 0.068	& 0.247 \\
& INHIBITION	& 0.062	& 0.072 \\
& JOB\_OR\_WORK	& 0.051	& 0.188 \\
& DISCREPANCY	& 0.045	& 0.289 \\
& SCHOOL	& 0.044	& 0.279 \\
\hline
\end{tabularx}
\caption{Top effective coefficient LIWC features over the model}
\label{tbl:topliwc}
\end{table}

\subsection{Summary}

Based on the result set we have, out initial data consisted of 1048
candidates and 86 LIWC features, before we build the multiple
regression model we have gone through different checks to eliminate
the features that may affect our model.

In the first assumption (Table~\ref{tbl:coefficmulti}), we removed 26
features because we detected its multicollinearity which will affect
the model, and also eliminated 102 candidates with critical values in
Mahalanobis distance check to reach the best model. On the model
summary (Table~\ref{tbl:modelsum}), R Square shows that 6.9\% accuracy
to predicate the output of the ``status'' with these
features. Signification (Table~\ref{tbl:anova}) of the model in ANOVA
should be under 0.05 but the value is 0.187 which means that the
current model cannot produce accurate prediction for the output. The
most effective feature in the model in Table~\ref{tbl:topliwc} are
NEGATIONS.


\section{Conclusions and Future Work}

We have started the analysis of interactions with our chosen
information system utilising a small set of features; there are many
more that we are planning to investigate. We list below the features
we have used in these experiments, and the remaining features that can
be used in future experiments:

\begin{itemize}
\item {\textbf{Data used in these experiments:}} Five Factors on text
  from Facebook Interaction; Five Factors on text from motivation
  letters; Status (e.g. Eligibility, Final Selections); Timeline/Footprint.
\item {\textbf{Data available but not used:}} Complete/detailed
  Timeline/Footprint; Complete/detailed Facebook profile which
  includes interactions between applicants; Demographic Information
  (e.g. gender, age); Appeals after receiving application decision
  (e.g. email or webpage help text); Five Factors Questionnaire (for
  current calls); Course Details; Course Favourites; Email
  interactions during application; Evaluation Grades (actual scores,
  broken down into categories); Additional forms (qualifications, CV,
  research proposal); Reasons for ineligibility (e.g. documents
  missing, already received scholarship within 12 months); Mobility
  Levels (e.g. undergraduate, postgraduate); Mobility Type
  (e.g. exchange, degree seeking); Host Institutions (where they applied).
\end{itemize}

The results from our two streams of experiments or explorations
provide a valuable initial indication about the data and an
appropriate way to represent and explore it. From Experiment 1, there
seems to be a strong relationship between the Five Factor feature
{\emph{Openness to Experience}} with a strong correlation with the
{\emph{Accepted}} group. The exploration of timeline behaviour is
dependent on our representation used for interactions, and the classes
derived. This is a first attempt at tackling this representation
problem. The same feature {\emph{Openness to Experience}} has no
group/class combinations that are significantly different than others.

The investigation using the Five Factors is very much in its initial
stages, and we wish to develop further to deliver an appropriate
instrument to the applicants to determine precisely their trait
values, and then we can explore how different personality types
interact. For this initial study we have utilised the only method
available to us, which is that determined by
\citeauthor{mairesse-et-al:2007}~\shortcite{mairesse-et-al:2007}. Our
future work will continue to focus in developing the multiple
regression model based upon more accurate data by including more
details about the user instead of only the motivation letter.

We are continuing with our dictionary-based approach to examine
content differences (LIWC), psycholinguistic differences (MRC), and
will also include affect differences using the Dictionary of Affect in
Language (DAL)~\cite{whissell:2008,whissell:2009}. The DAL is designed
to measure the emotional meaning of words and texts, comparing against
a word list rated by people for their activation, evaluation, and
imagery. Additionally we are using the
SentiWordNet~\cite{esuli+sebastiani:2006} and
WordNet-Affect~\cite{strapparava+valitutti:2004} extensions to
WordNet~\cite{miller:1995}. WordNet is a large lexical database of
English. Nouns, verbs, adjectives and adverbs are grouped into sets of
cognitive synonyms (synsets), each expressing a distinct
concept. Synsets are interlinked by means of conceptual-semantic and
lexical relations. SentiWordNet is a lexical resource for opinion
mining. WordNet-Affect labels synsets with a hierarchical set of
emotional categories.

We will correlate these with the Five Factors and trait emotional
intelligence~\cite{petrides+furnham:2001} through administration of
the Big Five Inventory (BFI, 44-item)~\cite{john-et-al:1991} and Trait
Emotional Intelligence Questionnaire--Short Form (TEIQue-SF, 30-item)
questionnaires designed to measure global trait emotional
intelligence~\cite{petrides+furnham:2006} respectively. In this way we
determine our own domain specific formulae for traits and behaviours,
instead of relying on the bias inherent in those from
\citeauthor{mairesse-et-al:2007}~\shortcite{mairesse-et-al:2007}. In
particular, we are especially interested in the temporal aspects of
traits and affects, which we plan to explore using Allen's logic of
time and action~\cite{allen:1983,allen:1984}.

\bibliographystyle{aaai}
\bibliography{icwsm15p1}

\end{document}
